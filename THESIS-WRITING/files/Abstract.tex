\begin{abstract}
	
	Traditional journaling provides individuals with a personal, distraction-free writing environment but cannot offer modern digital functionalities including searchability, content organization, and automated pattern recognition. In contrast, contemporary digital journaling applications deliver technical benefits yet frequently burden users with complex interface designs and feature overload, resulting in discontinued journaling habits. This research project develops \textbf{Collective}, a mobile journaling application designed to reconcile the benefits of traditional and digital journaling approaches.
	
	The main research objective focuses on developing a simplified journaling experience that preserves the essential characteristics of handwritten journaling while incorporating digital functionalities through background processing. The application presents a focused interface design enabling users to concentrate on composing individual entries with streamlined saving mechanisms. Natural language processing algorithms analyze entries automatically to create summaries, detect emotional patterns, and categorize content without requiring manual user input.
	
	The research methodology incorporates user-centered design principles, iterative development approaches, and natural language processing integration for automated content analysis. The system implementation utilizes Flutter framework for multi-platform deployment, Firebase for cloud infrastructure, and DeepSeek API for text processing and pattern identification. The application structure prioritizes performance and offline operation to maintain consistent user accessibility.
	
	Implemented functionalities include automated entry preservation, emotional pattern detection, content classification, search capabilities, and personalized insight generation. Data protection measures involve local processing and encrypted storage combined with optional cloud synchronization features.
	
	Evaluation results show measurable improvements in user participation and journaling consistency relative to conventional digital journaling solutions. Empirical studies reveal a 73\% increase in daily journaling activity and 85\% user satisfaction scores. The application addresses complexity barriers that commonly lead to discontinued use of digital journaling platforms.
	
	This research contributes to human-computer interaction by examining how automated analysis can improve user experiences without compromising interface simplicity. The study's importance relates to developing sustainable journaling solutions that accommodate users' emotional and organizational requirements while maintaining the personal, concentrated aspects that make traditional journaling effective. Future development directions include additional AI-driven analysis, social features, and mental health tracking integration.
	
\end{abstract}


\newpage

\begin{abstractmalay}

	Jurnal tradisional menawarkan pengalaman penulisan peribadi yang bebas gangguan tetapi kurang keupayaan digital moden seperti kebolehcarian, organisasi, dan pengecaman corak. Sebaliknya, aplikasi jurnal digital sedia ada menyediakan kelebihan teknikal tetapi sering membebankan pengguna dengan antara muka yang kompleks dan ciri-ciri berlebihan, yang membawa kepada pengabaian amalan jurnal. Projek ini menangani dikotomi asas ini dengan membangunkan \textbf{Collective}, sebuah aplikasi jurnal mudah alih yang merapatkan jurang antara kaedah jurnal tradisional dan digital.
	
	Objektif utama projek ini adalah untuk mencipta pengalaman jurnal minimalis yang mengekalkan kesederhanaan penulisan pen-dan-kertas sambil melaksanakan keupayaan digital secara bijak di belakang tabir. Aplikasi ini mempunyai antara muka yang diperkemas di mana pengguna hanya fokus pada menulis satu entri pada satu masa menggunakan gerak isyarat leret-untuk-simpan yang intuitif. Kecerdasan buatan memproses entri secara automatik untuk menghasilkan ringkasan, mengenal pasti corak emosi, dan mengatur kandungan tanpa memerlukan campur tangan pengguna.
	
	Metodologi yang digunakan termasuk prinsip reka bentuk berpusatkan pengguna, amalan pembangunan tangkas, dan integrasi algoritma pemprosesan bahasa semula jadi untuk analisis kandungan. Sistem ini dibina menggunakan rangka kerja Flutter untuk keserasian merentas platform, Firebase untuk perkhidmatan backend, dan model AI tersuai untuk analisis teks dan pengecaman corak.
	
	Ciri-ciri utama yang dilaksanakan termasuk penyimpanan automatik pintar, analisis corak emosi, pengkategorian kandungan automatik, fungsi carian, dan penjanaan wawasan peribadi. Sistem mengekalkan privasi data melalui pemprosesan tempatan dan storan yang disulitkan sambil menyediakan pilihan penyegerakan awan.
	
	Keputusan ujian menunjukkan peningkatan ketara dalam penglibatan pengguna dan konsistensi jurnal berbanding aplikasi jurnal digital tradisional. Kajian pengguna menunjukkan peningkatan 73\% dalam kekerapan jurnal harian dan penilaian kepuasan pengguna 85\%. Aplikasi ini berjaya menghapuskan halangan kerumitan yang biasanya menyebabkan pengguna meninggalkan platform jurnal digital.
	
	Projek ini menyumbang kepada bidang interaksi manusia-komputer dengan menunjukkan bagaimana kecerdasan buatan dapat meningkatkan pengalaman pengguna tanpa menjejaskan kesederhanaan. Kepentingannya terletak pada mencipta penyelesaian jurnal yang mampan yang menyesuaikan diri dengan keperluan emosi dan organisasi pengguna sambil mengekalkan pengalaman intim dan fokus jurnal tradisional.

\end{abstractmalay}

