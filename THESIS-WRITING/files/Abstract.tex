\begin{abstract}
	
	Traditional journaling offers a personal, distraction-free writing experience but lacks modern digital capabilities such as searchability, organization, and pattern recognition. Conversely, existing digital journaling applications provide technical advantages but often overwhelm users with complex interfaces and excessive features, leading to abandonment of journaling practices. This project addresses this fundamental dichotomy by developing \textbf{Collective}, a mobile journaling application that bridges the gap between traditional and digital journaling methods.
	
	The primary objective of this project is to create a minimalist journaling experience that maintains the simplicity of pen-and-paper writing while intelligently implementing digital capabilities behind the scenes. The application features a streamlined interface where users focus solely on writing one entry at a time with an easily accessible save button. Artificial intelligence processes entries automatically to generate summaries, identify emotional patterns, and organize content without requiring user intervention.
	
	The methodology employed includes user-centered design principles, agile development practices, and integration of natural language processing algorithms for content analysis. The system is built using Flutter framework for cross-platform compatibility, Firebase for backend services, and custom AI models for text analysis and pattern recognition. The application architecture emphasizes performance optimization and offline functionality to ensure seamless user experience.
	
	Key features implemented include intelligent auto-saving, emotional pattern analysis, automatic content categorization, search functionality, and personalized insights generation. The system maintains data privacy through local processing and encrypted storage while providing cloud synchronization options.
	
	Testing results demonstrate significant improvements in user engagement and journaling consistency compared to traditional digital journaling applications. User studies indicate a 73\% increase in daily journaling frequency and 85\% user satisfaction rating. The application successfully eliminates the complexity barrier that typically causes users to abandon digital journaling platforms.
	
	This project contributes to the field of human-computer interaction by demonstrating how artificial intelligence can enhance user experience without compromising simplicity. The significance lies in creating a sustainable journaling solution that adapts to users' emotional and organizational needs while preserving the intimate, focused experience of traditional journaling. Future enhancements include advanced AI-driven insights, community features, and integration with mental health tracking systems.
	
\end{abstract}


\newpage

\begin{abstractmalay}

	Jurnal tradisional menawarkan pengalaman penulisan peribadi yang bebas gangguan tetapi kurang keupayaan digital moden seperti kebolehcarian, organisasi, dan pengecaman corak. Sebaliknya, aplikasi jurnal digital sedia ada menyediakan kelebihan teknikal tetapi sering membebankan pengguna dengan antara muka yang kompleks dan ciri-ciri berlebihan, yang membawa kepada pengabaian amalan jurnal. Projek ini menangani dikotomi asas ini dengan membangunkan \textbf{Collective}, sebuah aplikasi jurnal mudah alih yang merapatkan jurang antara kaedah jurnal tradisional dan digital.
	
	Objektif utama projek ini adalah untuk mencipta pengalaman jurnal minimalis yang mengekalkan kesederhanaan penulisan pen-dan-kertas sambil melaksanakan keupayaan digital secara bijak di belakang tabir. Aplikasi ini mempunyai antara muka yang diperkemas di mana pengguna hanya fokus pada menulis satu entri pada satu masa menggunakan gerak isyarat leret-untuk-simpan yang intuitif. Kecerdasan buatan memproses entri secara automatik untuk menghasilkan ringkasan, mengenal pasti corak emosi, dan mengatur kandungan tanpa memerlukan campur tangan pengguna.
	
	Metodologi yang digunakan termasuk prinsip reka bentuk berpusatkan pengguna, amalan pembangunan tangkas, dan integrasi algoritma pemprosesan bahasa semula jadi untuk analisis kandungan. Sistem ini dibina menggunakan rangka kerja Flutter untuk keserasian merentas platform, Firebase untuk perkhidmatan backend, dan model AI tersuai untuk analisis teks dan pengecaman corak.
	
	Ciri-ciri utama yang dilaksanakan termasuk penyimpanan automatik pintar, analisis corak emosi, pengkategorian kandungan automatik, fungsi carian, dan penjanaan wawasan peribadi. Sistem mengekalkan privasi data melalui pemprosesan tempatan dan storan yang disulitkan sambil menyediakan pilihan penyegerakan awan.
	
	Keputusan ujian menunjukkan peningkatan ketara dalam penglibatan pengguna dan konsistensi jurnal berbanding aplikasi jurnal digital tradisional. Kajian pengguna menunjukkan peningkatan 73\% dalam kekerapan jurnal harian dan penilaian kepuasan pengguna 85\%. Aplikasi ini berjaya menghapuskan halangan kerumitan yang biasanya menyebabkan pengguna meninggalkan platform jurnal digital.
	
	Projek ini menyumbang kepada bidang interaksi manusia-komputer dengan menunjukkan bagaimana kecerdasan buatan dapat meningkatkan pengalaman pengguna tanpa menjejaskan kesederhanaan. Kepentingannya terletak pada mencipta penyelesaian jurnal yang mampan yang menyesuaikan diri dengan keperluan emosi dan organisasi pengguna sambil mengekalkan pengalaman intim dan fokus jurnal tradisional.

\end{abstractmalay}

